\documentclass[a4paper,11pt]{article}
%%%%%%%%%%%%%%%%%%%%%%%%%%%%%%%%%%%%%%%%%%%%%%%%%%%%%%%%%%%%%%%%%%%%%%%%%%%%%%%%%%%%%%%%%%%%%%%%%%%%%%%%%%%%%%%%%%%%%%%%%%%%
\usepackage[T1]{fontenc}
\usepackage[utf8]{inputenc}
\usepackage[dvipdfmx]{graphicx}
\usepackage{graphics,latexsym}
\usepackage{amsmath}
\usepackage{natbib}
\usepackage[dvips]{color}
\usepackage{subfigure}
\usepackage{verbatim}

\bibpunct{(}{)}{;}{a}{,}{,}

\textheight 24cm \textwidth 17cm \topmargin-2cm
%% \evensidemargin   -0.25cm
\oddsidemargin-0.2cm
%\pagestyle{empty}
\renewcommand{\baselinestretch}{1}

\begin{document}

\title{Abductive inference in bayesian networks}

\author{{David Quesada López}\\
{\small Computational Intelligence Group, Departamento de Inteligencia Artificial, Universidad Polit\'ecnica de Madrid, Spain}}

\date{}
\maketitle

%\title{}

%\address{}

\begin{abstract} Abductive inference in bayesian networks solves the problem of obtaining the most probable explanation (MPE) of a network given some evidence of its nodes. This inference can be total, if you aim to obtain the MPE of the whole network, or partial, if you are only interested in some of the nodes. In this state of the art we will cover both approaches and the methods used to solve them.
\end{abstract}


\ \\
KEY WORDS: Bayesian networks; Abductive inference; Approximate inference




\section{Introduction}

About the problem.



\section{State-of-the-art}

Describe what others have done, citing their works. There are different formats: a paper as \cite{uncu2007}, a book as \cite{hosmer-lemeshow2000}, or a book chapter as \cite{wold75} or as in the proceedings of a conference (\cite{shakhnarovich2001icml}).

    \subsection{More specific}


Perhaps some subsections are needed.

\section{Conclusions and future research}

What are the main open lines for research.



\bibliographystyle{plainnat}
\begin{thebibliography}{}

\bibitem[De Campos \textit{et al.}, 1999]{deCampos1999}
De Campos L. M., Gámez J. A., Moral S. (1999). Partial abductive inference in Bayesian belief networks using a genetic algorithm. Pattern Recognition Letters, 20(11), 1211-1217.

\bibitem[De Campos \textit{et al.}, 2002]{deCampos2002}
De Campos L. M., Gamez J. A., Moral S. (2002). Partial abductive inference in Bayesian belief networks-an evolutionary computation approach by using problem-specific genetic operators. IEEE Transactions on Evolutionary Computation, 6(2), 105-131.

\bibitem[Fortier \textit{et al.}, 2013]{fortier2013}
Fortier N., Sheppard J., Pillai K. G. (2013, April). Bayesian abductive inference using overlapping swarm intelligence. In Swarm Intelligence (SIS), 2013 IEEE Symposium on (pp. 263-270). IEEE.

\bibitem[Gelsema, 1995]{gelsema1995}
Gelsema E. S. (1995). Abductive reasoning in Bayesian belief networks using a genetic algorithm. Pattern Recognition Letters, 16(8), 865-871.

\end{thebibliography}


\end{document}
