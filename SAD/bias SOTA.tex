\documentclass[a4paper,11pt]{article}
%%%%%%%%%%%%%%%%%%%%%%%%%%%%%%%%%%%%%%%%%%%%%%%%%%%%%%%%%%%%%%%%%%%%%%%%%%%%%%%%%%%%%%%%%%%%%%%%%%%%%%%%%%%%%%%%%%%%%%%%%%%%
\usepackage[T1]{fontenc}
\usepackage[utf8]{inputenc}
\usepackage[dvipdfmx]{graphicx}
\usepackage{graphics,latexsym}
\usepackage{amsmath}
\usepackage{natbib}
\usepackage[dvips]{color}
\usepackage{subfigure}
\usepackage{verbatim}

\bibpunct{(}{)}{;}{a}{,}{,}

\textheight 24cm \textwidth 17cm \topmargin-2cm
%% \evensidemargin   -0.25cm
\oddsidemargin-0.2cm
%\pagestyle{empty}
\renewcommand{\baselinestretch}{1}

\begin{document}

\title{Sesgos en la toma de decisiones}

\author{{David Quesada López}\\
{\small Computational Intelligence Group, Departamento de Inteligencia Artificial, Universidad Polit\'ecnica de Madrid, Spain}}

\date{}
\maketitle

%\title{}

%\address{}

\begin{abstract} 
Cuando nos enfrentamos al problema de modelar el riesgo y la incertidumbre en un sistema de decisión, aparecen tanto en nuestro sistema como en los expertos en los que lo basamos sesgos que afectan a la toma de decisiones y que degeneran su calidad. Estos sesgos pueden ser cognitivos o motivacionales, y afectan de manera diferente a un decisor individual o a un conjunto de decisores. En este estado del arte, trataremos sesgos de estos dos tipos y métodos para evitarlos.
\end{abstract}


\ \\
KEY WORDS: Cognitive bias; Motivational bias; Decision making;




\section{Introducción}

A la hora de realizar decisiones, los seres humanos pueden sufrir sesgos. Entendemos por sesgo una distorsión o simplificación en el juicio de un ser humano a la hora de tomar una decisión, lo que provoca que la decisión que tomemos no sea la óptima (\cite{toet2016}). Estos sesgos son propios de los expertos a los que recurrimos a la hora de modelar sistemas de decisión, y los sesgos que afectan a sus juicios se pueden trasladar a nuestro modelo, reduciendo la calidad de sus decisiones.

Los sesgos pueden dividirse en individuales, si la toma de decisiones la hace un sólo sistema o experto, o grupales, cuando afectan a un conjunto de decisores en consenso. A su vez, los sesgos individuales se subdividen en cognitivos o motivacionales en función de qué es lo que los origina.

Los sesgos cognitivos son discrepancias sistemáticas entre la decisión óptima o de más utilidad y la elegida finalmente por el decisor (\cite{von1986}). Este tipo de sesgos han sido los más tratados en la literatura desde sus primeras menciones (\cite{tversky1974}). Un ejemplo conocido de este tipo de sesgos es el anclaje, que tiende a quedarse en los primeros valores generados al estimar una variable. En general, no han aparecido muchos sesgos cognitivos recientes, pero se cuenta con algunas tecnicas heurísticas de reducción del sesgo o \textit{debiasing}.

En contraposición, los sesgos motivacionales son debidos al deseo o la adversión del decisor a eventos, consecuencias, resultados o elecciones (\cite{kunda1990}). Este tipo de sesgos, más reciente y menos estudiado por la literatura, son más difíciles de corregir que los cognitivos. Ejemplo de este tipo de sesgos sería el sesgo de confirmación, que hace que el experto tienda a tomar decisiones que refuercen sus creencias iniciales.

Hay también varios casos concretos de sesgos grupales, pero uno de los mayores problemas de los decisores grupales es que sufren de sesgos individuales aumentados (\cite{kerr2004}).

En este estado del arte, se tratará de dar una visión de los sesgos cognitivos y motivacionales más importantes y la forma de paliarlos. No se tratará el tema de los sesgos grupales, ya que pueden reducirse en cierto grado tratando los sesgos individuales y que los métodos de reducción de sesgos grupales no son especialmente fiables, siendo casi común el uso de expertos con diferentes puntos de vista como método de \textit{debiasing}.


\section{State-of-the-art}
El estudio de los sesgos se inició con \cite{tversky1974}, donde se identificó a los sesgos como una fuente de errores sistemáticos en la toma de decisiones que había que corregir. Desde entonces, se ha tratado de encontrar nuevos sesgos y maneras de contrarrestar sus efectos. Los métodos seguidos para demostrar sesgos son, en el mejor de los casos, empíricos por medio de experimentos sociales, como en el caso de \cite{moore2008}. Una vez reconocido un sesgo, se tratan de dar remedios para él, pero no es muy común encontrar bases más allá de la heurística que soporten esas medidas. En las siguientes secciones se listarán una serie de sesgos relevantes a tener en cuenta a la hora de consultar con expertos para la toma de decisiones.

\subsection{Sesgos cognitivos}
Los sesgos cognitivos son los que se tratan de manera más extensa en la literatura. El grado en el que afectan a la toma de decisiones es muy variable, pero los métodos de \textit{debiasing} para afrontarlos han sido probados con los años. Algunos de los más relevantes son:

\begin{itemize}
\item El \textbf{exceso de seguridad} o \textit{overconfidence} (\cite{moore2008}). Este sesgo ocurre cuando el decisor sobreestima su capacidad para estimar valores o cuando las estimaciones que hace son con demasiada precisión. En el primer caso, el decisor les da mayor peso del que realmente tienen a esos valores. En el segundo caso, a la hora de estimar un valor el decisor no se saldrá de un intervalo muy reducido, creyendo excesivamente que la aproximación tiene que pertenecer a ese intervalo y no contemplando los casos en los que eso no ocurra. 
Este es uno de los sesgos que más se ha estudiado en la literatura y cuyos efectos son más severos a la hora de la toma de decisiones, generando decisiones erróneas basándose en información incorrecta sobre la capacidad predictiva propia. Por esto, los métodos de \textit{debiasing} para resolverlo pasan por aportar información sobre la calidad predictora para que tengan menos peso las propias estimaciones, o por proponer casos de valores extremos, para que el decisor compruebe si estos valores fuera del intervalo son viables o no.

\item El \textbf{anclaje}. Esto ocurre cuando hay una gran influencia en los decisores para hacer estimaciones cercanas a un valor inicial proporcionado (\cite{chapman2002}). En la práctica, cuando no se tiene una idea clara de qué valor puede tomar una variable y se nos proporciona un posible valor para ella, se tiende a aproximar ese valor a algo cercano a este ancla, pensando que ese es el valor correcto. El anclado puede ser beneficioso si este ancla es un valor fiable, pero si este valor es arbitrario o no muy seguro estaremos incurriendo en errores de decisión severos. Para prevenir un anclaje erróneo, la práctica más común es dar varias anclas en valores extremos opuestos, para que el decisor no de tanta relevancia al ancla inicial y pueda explorar valores intermedios.

\item La \textbf{facilidad de recordar}. Este sesgo hace que los expertos estimen la probabilidad de ocurrencia de un suceso en base a lo fácil que es de recordar un caso concreto. Esto hace que sucesos que son fáciles de recordar se les estime erróneamente como más probables (\cite{buon2008}). Para evitar este sesgo hay que proporcionar al decisor contraejemplos de casos en los que no es tan probable el suceso a estimar, o si se tienen estadísticas sobre su ocurrencia.

\item El sesgo de \textbf{ganancias y pérdidas}. En él, la manera de presentar una pregunta como de ganancia o de pérdida puede sesgar la respuesta obtenida (\cite{tversky1981}). Cuando se presenta como una ganancia, el efecto es tender a elegir esa decisión, mientras que si se presenta como una pérdida, se tiende a evitarla a pesar de que ambas representan la misma decisión en las mismas condiciones. Esto da lugar al efecto conocido como \textit{framing}. Para evitar este sesgo, hay que presentar al decisor la igualdad de ambos casos o calcular las utilidades de ambas formas de presentar la pregunta y comprobar que sean diferentes, para desestimar

\end{itemize}

Existen más sesgos cognitivos, como el de representación miope, el de omisión de variables relevantes o el efecto de certidumbre. Por motivos de espacio, no se entra a tratar todos ellos en concreto.
\subsection{Sesgos motivacionales}
Los sesgos motivacionales no han sido tan estudiados en la literatura, pero su efecto es tan relevante como el de los cognitivos. Son en general más complicados de resolver, y quien los comete no tiene por qué ser consciente de su inclinación hacia ellos. Algunos de los más conocidos son:

\begin{itemize}

\item El sesgo de \textbf{confirmación}. Ocurre cuando un experto busca confirmar una idea previa que tiene, esquivando activamente argumentos en contra y buscando argumentos a favor, siendo selectivo con la información usada (\cite{correia2011}). Para atenuar este sesgo, se puede confrontar la opinion de dos expertos con diferentes puntos de vista, para que argumentos a favor y en contra estén presentes, o se puede tratar de pedir al experto que explique los puntos a favor y en contra de su razonamiento, para que se fuerze a ver los argumentos en contra que no está teniendo en cuenta. El caso contrario, el sesgo de desconfirmación, afecta a la toma de decisiones cuando un experto tiene opiniones muy adversas sobre un suceso (\cite{taber2009}).

\item El sesgo por el \textbf{deseo} de la ocurrencia de un suceso. Cuando para el decisor un suceso es muy deseable que ocurra, se le considera una mayor probabilidad de que suceda de la que realmente tiene a la hora de predecirlo (\cite{krizan2007}). También existe su versión opuesta, donde la adversión a un suceso hace que se considere menor su probabilidad de ocurrir. Para afrontar estos sesgos se puede usar la opinion de varios expertos con diferentes puntos de vista, o se puede intentar hacer una predicción en contra del suceso que se desea, para que el experto se enfrente a si de verdad es poco probable que no suceda.

\item El sesgo por la \textit{preferencia} de un resultado concreto. Cuando se está emocionalmente predispuesto a un resultado antes que a otro nuestra capacidad para tomar decisiones puede verse afectada (\cite{slovic2004}). Un ejemplo claro de esto sería a la hora de contratar un seguro, si prefiriésemos asegurar objetos de no mucho valor monetario pero sí emocional antes que objetos muy caros a los que no tenemos mucho apego. Para evitar este sesgo, hay que comprobar que el \textit{framing} de las preguntas no sesgue emocionalmente a los decisores y consultar a expertos con diferentes puntos de vista sobre el tema.

\end{itemize}

\section{Conclusiones y líneas de investigación abiertas}

El estudio de los sesgos ayuda en gran medida a obtener una mejor calidad en la toma de decisiones. Conocer diferentes tipos de sesgos y maneras de corregirlos ayuda a prevenir errores a la hora de modelar los decisores. Sin embargo, muchas de las técnicas de \textit{debiasing} son puramente heurísticas y no está demostrado su funcionamiento, y los sesgos estudiados actualmente son los identificados al principio por \cite{tversky1974}, si son cognitivos, y por \cite{kunda1990}, si son motivacionales.

En este área, la línea de investigación más antigua y prolífera se centra en el descubrimiento de sesgos y la demostración de sus efectos de manera empírica. Recientemente, el estudio de los sesgos se ha centrado más en recopilar todos los ya conocidos más que en encontrar nuevos. 

Hay mucha más literatura sobre los sesgos cognitivos, mientras que no hay tantos estudios de sesgos motivacionales. Es posible que si se investigue en ese ámbito se encuentren nuevos sesgos de este tipo.

Otra línea de investigación poco explorada es la demostración empírica del funcionamiento de las técnicas de \textit{debiasing}, como en el caso de investigaciones recientes como la de \cite{montibeller2015}.



\bibliographystyle{plainnat}
\begin{thebibliography}{}

\bibitem[Buontempo and Brockner, 2008]{buon2008}
Buontempo G., Brockner J. (2008). Emotional Intelligence and the Ease of Recall Judgment Bias: The Mediating Effect of Private Self‐Focused Attention. Journal of Applied Social Psychology, 38(1), 159-172.

\bibitem[Chapman and Johnson, 2002]{chapman2002}
Chapman G. B., Johnson, E. J. (2002). Incorporating the irrelevant: Anchors in judgments of belief and value. Heuristics and biases: The psychology of intuitive judgment, 120-138.

\bibitem[Correia, 2011]{correia2011}
Correia V. (2011). Biases and fallacies: The role of motivated irrationality in fallacious reasoning. Cogency, 3(1), 107-126.

\bibitem[Finucane \textit{et al.}, 2000]{finu2000}
Finucane M. L., Alhakami A., Slovic P., Johnson S. M. (2000). The affect heuristic in judgments of risks and benefits. Journal of behavioral decision making, 13(1), 1.

\bibitem[Kerr and Tindale, 2004]{kerr2004}
Kerr NL, Tindale RS (2004) Group performance and decision making. Annu Rev Psychol 55:623–655. doi:10.1146/annurev.psych.55.090902.142009

\bibitem[Krizan and Windschitl, 2007]{krizan2007}
Krizan Z., Windschitl P. D. (2007). The influence of outcome desirability on optimism. Psychological bulletin, 133(1), 95.

\bibitem[Kunda, 1990]{kunda1990}
Kunda Z (1990) The case for motivated reasoning. Psychol Bull 108:480–498. doi:10.1037/0033-2909.108.3.480

\bibitem[Montibeller and von Winterfeldt, 2015]{montibeller2015}
Montibeller G., von Winterfeldt D. (2015). Biases and debiasing in multi-criteria decision analysis. In System Sciences (HICSS), 2015 48th Hawaii International Conference on (pp. 1218-1226). IEEE.

\bibitem[Moore \textit{et al}, 2008]{moore2008}
Moore, Don A., Paul J. Healy (2008) The Trouble with Overconfidence, Psychological Review, 115, 502–517.

\bibitem[Slovic \textit{et al.}, 2004]{slovic2004}
Slovic P., Finucane M. L., Peters E., MacGregor D. G. (2004). Risk as analysis and risk as feelings: Some thoughts about affect, reason, risk, and rationality. Risk analysis, 24(2), 311-322.

\bibitem[Taber \textit{et al.}, 2009]{taber2009}
Taber C. S., Cann D., Kucsova S. (2009). The motivated processing of political arguments. Political Behavior, 31(2), 137-155.

\bibitem[Toet \textit{et al.}, 2016]{toet2016}
Toet A., Anne-Marie Brouwer, Karel van den Bosch, J.E. (Hans) Korteling (2016) Effects of personal characteristics on susceptibility to decision bias: a literature study. International Journal of Humanities and Social Sciences \textit{Vol. 8 No. 5, pp. 1-17}

\bibitem[Tversky and Kahneman, 1974]{tversky1974}
Tversky A, Kahneman D (1974) Judgment under uncertainty: heuristics and biases. Science
185:1124–1131. doi:10.1126/science.185.4157.1124

\bibitem[Tversky and Kahneman, 1981]{tversky1981}
Tversky A., Kahneman D. (1981). The framing of decisions and the psychology of choice. Science, 211(4481), 453-458.

\bibitem[von Winterfeldt and Edwards, 1986]{von1986}
von Winterfeldt D, Edwards W (1986) Decision analysis and behavioral research. Cambridge University Press, New York, NY



\end{thebibliography}


\end{document}
