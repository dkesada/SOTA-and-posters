\documentclass[a4paper,11pt]{article}
%%%%%%%%%%%%%%%%%%%%%%%%%%%%%%%%%%%%%%%%%%%%%%%%%%%%%%%%%%%%%%%%%%%%%%%%%%%%%%%%%%%%%%%%%%%%%%%%%%%%%%%%%%%%%%%%%%%%%%%%%%%%
\usepackage[T1]{fontenc}
\usepackage[utf8]{inputenc}
\usepackage[dvipdfmx]{graphicx}
\usepackage{graphics,latexsym}
\usepackage{amsmath}
\usepackage{natbib}
\usepackage[dvips]{color}
\usepackage{subfigure}
\usepackage{verbatim}

\bibpunct{(}{)}{;}{a}{,}{,}

\textheight 24cm \textwidth 17cm \topmargin-2cm
%% \evensidemargin   -0.25cm
\oddsidemargin-0.2cm
%\pagestyle{empty}
\renewcommand{\baselinestretch}{1}

\begin{document}

\title{Sesgos en la toma de decisiones}

\author{{David Quesada López}\\
{\small Computational Intelligence Group, Departamento de Inteligencia Artificial, Universidad Polit\'ecnica de Madrid, Spain}}

\date{}
\maketitle

%\title{}

%\address{}

\begin{abstract} 
Cuando nos enfrentamos al problema de modelar el riesgo y la incertidumbre en un sistema de decisión, aparecen tanto en nuestro sistema como en los expertos en los que lo basamos sesgos que afectan a la toma de decisiones y que degeneran su calidad. Estos sesgos pueden ser cognitivos o motivacionales, y afectan de manera diferente a un decisor individual o a un conjunto de decisores. En este estado del arte, trataremos sesgos de estos dos tipos y métodos para evitarlos.
\end{abstract}


\ \\
KEY WORDS: Cognitive bias; Motivational bias; Decision making;




\section{Introducción}

A la hora de realizar decisiones, los seres humanos pueden sufrir sesgos. Entendemos por sesgo una distorsión o simplificación en el juicio de un ser humano a la hora de tomar una decisión, lo que provoca que la decisión que tomemos no sea la óptima (\cite{toet2016}).



\section{State-of-the-art}


\subsection{More specific}


Perhaps some subsections are needed.

\section{Conclusions and future research}

What are the main open lines for research.



\bibliographystyle{plainnat}
\begin{thebibliography}{}

\bibitem[Toet \textit{et al.}, 2016]{toet2016}
Alexander Toet, Anne-Marie Brouwer, Karel van den Bosch, J.E. (Hans) Korteling (2016) Effects of personal characteristics on susceptibility to decision bias: a literature study. International Journal of Humanities and Social Sciences \textit{Vol. 8 No. 5, pp. 1-17}

\bibitem[Montibeller and Winterfeldt, 2018]{montibeller2018}
Montibeller G., von Winterfeldt D. (2018) Individual and Group Biases in Value and Uncertainty Judgments. In: Dias L., Morton A., Quigley J. (eds) Elicitation. \textit{International Series in Operations Research \& Management Science, vol 261. Springer, Cham}

\bibitem[Uncu and T\"urksen, 2007]{uncu2007}
Uncu, O. and T\"urksen, I.B. (2007) A novel feature selection
approach: Combining feature wrappers and filters. \textit{Information Sciences}, 177, 449--466.

\bibitem[Wold, 1975]{wold75} Wold, H. (1975) Soft modelling by latent variables: The non-linear iterative partial least squares ({NIPALS})
approach. In Gani, J. (ed), \textit{Perspectives in Probability and Statistics}, Academic Press, London, pp. 117--142.


\end{thebibliography}


\end{document}
