\documentclass[a4paper,11pt]{article}
%%%%%%%%%%%%%%%%%%%%%%%%%%%%%%%%%%%%%%%%%%%%%%%%%%%%%%%%%%%%%%%%%%%%%%%%%%%%%%%%%%%%%%%%%%%%%%%%%%%%%%%%%%%%%%%%%%%%%%%%%%%%
\usepackage[T1]{fontenc}
\usepackage[utf8]{inputenc}
\usepackage[dvipdfmx]{graphicx}
\usepackage{graphics,latexsym}
\usepackage{amsmath}
\usepackage{natbib}
\usepackage[dvips]{color}
\usepackage{subfigure}
\usepackage{verbatim}

\bibpunct{(}{)}{;}{a}{,}{,}

\textheight 24cm \textwidth 17cm \topmargin-2cm
%% \evensidemargin   -0.25cm
\oddsidemargin-0.2cm
%\pagestyle{empty}
\renewcommand{\baselinestretch}{1}

\begin{document}

\title{Selección del subconjunto de variables y clustering}

\author{{David Quesada López}\\
{\small Computational Intelligence Group, Departamento de Inteligencia Artificial, Universidad Polit\'ecnica de Madrid, Spain}}

\date{}
\maketitle

%\title{}

%\address{}

\begin{abstract} La selección del subconjunto de variables es el proceso de encontrar un subconjunto de las variables originales que genere los mejores resultados comparado con todo el conjunto de variables en términos de la eficiencia y la efectividad. Los métodos de selección del subconjunto de variables están divididos en tres categorías dependiendo del criterio de evaluación: métodos filter, wrapper o embebidos. En este estado del arte, revisaremos por encima estas tres variantes y nos centraremos en los métodos de filtrado que utilizan clustering.
\end{abstract}


\ \\
KEY WORDS: Feature subset selection; Filter method; Feature clustering;




\section{Introducción}

Cuando te enfrentas a un problema de aprendizaje automático, uno de los inconvenientes que puedes tener es un gran número de variables en tu dataset. De estas variables no todas tienen por qué ser relevantes para tu problema, y en algunos casos usar todo el conjunto puede disminuir la efectividad de nuestro sistema (\cite{hliu2011}).

El problema de elegir el subconjunto de variables óptimo es un problema NP-completo (\cite{albrecht1982}), por lo que el objetivo es encontrar el subconjunto de las variables que nos satisfaga en términos de efectividad y eficiencia, no el óptimo. Cuando hablamos de efectividad nos referimos a la precisión final de nuestro sistema que obtenemos a partir del subconjunto de las variables, y al hablar de eficiencia estamos hablando sobre el tiempo que tardamos en obtener dicho subconjunto.

Para atacar este problema se distinguen tres puntos de vista diferentes en base al criterio seguido para obtener el subconjunto de variables útiles:

\begin{itemize}
 
\item Si la selección de este subconjunto está basada en la precisión predictiva de nuestro algoritmo de aprendizaje resultante, entonces nos encontramos ante un método wrapper. En este caso, las variables van siendo tenidas en cuenta en el modelo si mejoran su capacidad predictiva. El modelo es usado como una caja negra para ver su capacidad predictiva. 
Los algoritmos wrapper (\cite{kohavi1997}) son bastante eficaces en cuanto a precisión del modelo resultante, sin embargo sufren de una carga computacional muy elevada que los hace inabarcables en datasets con un conjunto muy grande de variables.
\item Si la selección del subconjunto es independiente del algoritmo de aprendizaje usado después, entonces hablamos de un método de filtrado o filter. Estos métodos se basan en medidas matemáticas como la información mutua o el coeficiente de correlación de Pearson para relacionar unas variables con otras y medir su relevancia (\cite{gheyas2009}). Estos métodos no son muy costosos computacionalmente comparados con los métodos wrapper y no están ligados al algoritmo de aprendizaje usado después, sin embargo no garantizan la precisión del modelo final (\cite{fast}).
\item Si se usa la selección de variables como parte del algoritmo de aprendizaje, entonces se habla de métodos embebidos o embedded. Ejemplos clásicos de estos métodos son los árboles de decisión o las redes neuronales (\cite{breiman1984}).  
\item También existen los llamados métodos híbridos, que combinan la rapidez de los métodos de filtrado con los buenos resultados de los métodos wrapper (\cite{uncu2007}).
\end{itemize} 

Este problema es también dividido a veces en supervisado, si tenemos algunas etiquetas que indiquen qué variables son útiles, o no supervisado, si no tenemos ninguna indicación de la utilidad de las variables (\cite{li2014}).

En este estado del arte nos centraremos en los métodos de filtrado. El resto del documento se organiza en la sección 2, donde se repasan distintos métodos de filtrado y las técnicas que se utilizan, y la sección 3, donde se habla de las líneas de investigación abiertas en este ámbito.

\section{State-of-the-art}

Al elegir un subconjunto de variables, tenemos que fijarnos en que estas sean relevantes y no redundantes entre sí. Una variable es relevante cuando ayuda a discernir más la clase de una instancia que otra variable, y es redundate cuando aporta la misma información sobre la clase a la que se pertenece que otra variable. (\cite{fast})
Que una variable no sea relevante provoca empeoramientos en la precisión de nuestro algoritmo de aprendizaje (\cite{john1994}), y la redundancia entre variables provoca deterioros en el rendimiento de nuestro modelo, por lo que conviene evitar ambos casos.

Los primeros métodos que aparecieron para el filtrado se centraban en encontrar variables relevantes para el modelo, uno de los ejemplos más clásicos es el algoritmo \textit{Relief} (\cite{kira1992}). \textit{Relief} discrimina unas variables de otras dependiendo de lo bien que diferencian la clase según un criterio basado en análisis estadístico de la distancia. 

Como primera aproximación, es un buen modo de reducir en parte la dimensionalidad del conjunto de variables, pero este algoritmo no comprueba que las variables que selecciona estén muy correlacionadas entre sí, por lo que puede seleccionar variables redundantes entre sí que no ayuden a mejorar la precisión del modelo y sin embargo dejar fuera variables que por sí mismas no están muy correlacionadas con la clase pero que aportan más información en conjunto con las que tenemos.

Para solventar el problema de la redundancia entre variables se crean nuevos algoritmos como \textit{CFS} y \textit{FCBF} que emplean medidas como la información mutua y la incertidumbre simétrica para comprobar la redundancia entre variables \cite{yu2004}

    \subsection{More specific}


Perhaps some subsections are needed.

\section{Conclusions and future research}

What are the main open lines for research.



\bibliographystyle{plainnat}
\begin{thebibliography}{}

\bibitem[Uncu and T\"urksen, 2007]{uncu2007}
Uncu, O. and T\"urksen, I.B. (2007) A novel feature selection
approach: Combining feature wrappers and filters. \textit{Information Sciences}, 177, 449--466.

\bibitem[H. Liu X. \textit{et al.}, 2011]{hliu2011}
H. Liu, X. Wu, and S. Zhang, Feature selection using hierarchical feature clustering, in \textit{Proc. ACM Int. Conf. Inform. Knowl. Manage.}, New York, NY, USA, 2011.

\bibitem[A.A. Albrecht, 1982]{albrecht1982}
A.A. Albrecht, Stochastic local search for the feature set problem, with applications to microarray data, Applied Mathematics and Computation 183 (2006) 1148–1164

\bibitem[Kohavi and John, 1997]{kohavi1997}
R. Kohavi and G. John. Wrappers for feature selection. \textit{Artificial Intelligence}, 97(1-2):273–324,
December 1997.

\bibitem[Gheyas and Smith, 2009]{gheyas2009}
I.A. Gheyas, L.S. Smith, Feature subset selection in large dimensionality domains, Pattern Recognition (2009), doi:
10.1016/j.patcog.2009.06.009.

\bibitem[Song \textit{el al.}, 2013]{fast}
Qinbao Song, Jingjie Ni and Guangtao Wang, A fast clustering-based feature subset selection algorithm for high dimensional data. \textit{IEEE TRANSACTIONS ON KNOWLEDGE AND DATA ENGINEERING VOL:25 NO:1} year 2013.

\bibitem[Breiman \textit{el al.}, 1984]{breiman1984}
L. Breiman, J. H. Friedman, R. A. Olshen, and C. J. Stone. Classification and Regression Trees. Wadsworth and Brooks, 1984.

\bibitem[Li \textit{el al.}, 2014]{li2014}
Zechao Li, Jing Liu, Yi Yang, Xiaofang Zhou,and Hanqing Lu, Clustering-guided sparse structural learning
for unsupervised feature selection. \textit{IEEE TRANSACTIONS ON KNOWLEDGE AND DATA ENGINEERING VOL:26 NO:9} September 2014.

\bibitem[John Kohavi and Pfleger, 1994]{john1994}
John G.H., Kohavi R. and Pfleger K., Irrelevant Features and the Subset Selection Problem, In the Proceedings of the Eleventh International Conference on Machine Learning, pp 121-129, 1994.

\bibitem[Kira and Rendell, 1992]{kira1992}
Kenji Kira, Larry A. Rendell. A practical approach to feature selection, 1992.

\bibitem[Yu and Liu, 2004]{yu2004}
L. Yu, H. Liu, Efficient Feature Selection via Analysis
of Relevance and Redundancy. Journal of Machine
Learning Research 5: 1205-1224, 2004.

\end{thebibliography}


\end{document}
